%BSP Bild
%\begin{figure}[h]
%\centering
%\includegraphics[width=7cm,height=2.8cm]{Fotos/logo.jpg}
%\end{figure}

%BSP Sektion
%\section{Test}
%\subsection{Umsetzung}
%\subsubsection{Eingabe}

%BSP Aufzählung
%\begin{enumerate}
%\item Wie realisieren wir den Roboter mechanisch?
%\item Wie lässt sich das Einlesen des Sudokus umsetzen?
%\item Wie implementieren wir den Algorithmus?
%\end{enumerate}

%BSP Aufzählung
%\begin{itemize}
%\item Helligkeitssensor
%\item Farbsensor
%\end{itemize}


%-------------------
%Beginn des Kopfbereiches
%-------------------

%Wir verwenden eine DIN-A4-Seite und die Schriftgrš§e 12.
\documentclass[a4paper,fontsize=13pt]{scrartcl} 

 \usepackage{float}

%ein Versuch die Silbentrennung auszuschalten
%\hyphenpenalty=10000
%\exhyphenpenalty=10000
\usepackage[none]{hyphenat} 
\sloppy

%Fuer Abbildungsverzeichnis
\usepackage{graphicx}

%Diese drei Pakete benštigen wir fŸr die Umlaute, Deutsche Silbentrennung etc.
\usepackage[utf8x]{inputenc}
\usepackage[ngerman]{babel}
\usepackage[T1]{fontenc}

%Für Zeilenabstände
\usepackage{setspace}

% Für Bilder 
\usepackage{graphicx}

%sorgt dafür, dass deutsche Anführungszeichen in Zitaten
\usepackage[babel,german=quotes]{csquotes}

%Das Paket erzeugt ein anklickbares Verzeichnis in der PDF-Datei.
\usepackage{hyperref}

\usepackage{titleref}

%für Eurosymbole
\usepackage{eurosym}

\usepackage{enumitem}


% Stil der Zitate und der Bibliographie
\bibliographystyle{unsrt}

% Einbindung von Quellcode
\usepackage{listings}  


%Das Paket wird fŸr die anderthalb-zeiligen Zeilenabstand benštigt
\usepackage{setspace}

%Fuer Abbildungsverz im Inhaltsverz
\usepackage{tocbibind}

\usepackage{pdfpages}

\usepackage{amssymb}
\usepackage{amsmath}
\usepackage{amsthm}

%Für Bögen über Strecken
\usepackage{arcs}

% Einbindung von Quellcode

\usepackage{xcolor}
\usepackage{listings}
\lstset{
	numbers=left, 
	numberstyle=\small, 
	numbersep=8pt,
	language=JAVA, 
	frame = single, 
	framexleftmargin=15pt}

%EinrŸckung eines neuen Absatzes
\setlength{\parindent}{0em}

%Definition der Ränder
\usepackage[paper=a4paper,left=30mm,right=30mm,top=30mm,bottom=30mm]{geometry} 

%Abstand der Fußnoten
%\deffootnote{1em}{1em}{\textsuperscript{\thefootnotemark\ }}

%Regeln, bis zu welcher Tiefe (section,subsection,subsubsection) †berschriften angezeigt werden sollen (Anzeige der †berschriften im Verzeichnis / Anzeige der Nummerierung)
%\setcounter{tocdepth}{3}
%\setcounter{secnumdepth}{3}

%-------------------
%Ende des Kopfbereiches
%-------------------

\usepackage{contour}
\usepackage{ulem}

\renewcommand{\ULdepth}{1.8pt}
\contourlength{0.8pt}

\newcommand{\myuline}[1]{%
  \uline{\phantom{#1}}%
  \llap{\contour{white}{#1}}%
}

\setcounter{tocdepth}{4}
\setcounter{secnumdepth}{4}

\begin{document}

\begin{titlepage}

\begin{center}


% Upper part of the page
\includegraphics[width=0.5\textwidth]{pictures/lmg8.png}\\[1.5cm]

%\textsc{\LARGE University of Beer}\\[1.5cm]

\textsc{\Large Dokumentation der Projektarbeit }\\[0.4cm]

\textsc{Leitung: Frau Müller}\\[0.4cm]


% Title
\noindent\rule{1\textwidth}{0.75pt}
%scheint die Linie zu machen
%es gibt medskip bigskip und smallskip
\vspace{0.4cm}


{ \huge \bfseries Role-Play-Game}\\[0.3cm]
\vspace{0.1cm}
\noindent\rule{1\textwidth}{0.75pt}
\vspace{0.4cm}


% Author and supervisor
\begin{minipage}{0.4\textwidth}
\begin{flushleft} \large


\end{flushleft}
\end{minipage}
\hfill
\begin{minipage}{0.5\textwidth}
\begin{flushright} \large
\vspace{1cm}
Lukas Fausten, Jonas Dalchow, Philipp Maleri, Philipp Neumer, Tim Holzenkamp \\
\vspace{0.5cm}
Informatik Leistungskurs 12 MLL\\[0.8cm]




\end{flushright}
\end{minipage}

\vfill

% Bottom of the page
{\large \today} 



\end{center}

\end{titlepage}

\newpage

\tableofcontents

\newpage

\section{Lastenheft}
\label{Lastenheft}

\textbf{UR001} \\
\textbf{Aussage} Das Role-Play-Game soll eine graphische Benutzeroberfläche haben. \\
\textbf{Priorität A} \\

\textbf{UR002} \\
\textbf{Aussage} Der Spieler soll einen Helden erstellen bzw. steuern können. \\
\textbf{Priorität A} \\

\textbf{UR003} \\
\textbf{Aussage} Der erstellte Held soll in verschiedenen Runden gegen Monster kämpfen. \\
\textbf{Priorität A} \\

\textbf{UR004} \\
\textbf{Aussage} Es soll eine Kampfanimation geben. \\
\textbf{Priorität A} \\

\textbf{UR005} \\
\textbf{Aussage} Es soll verschiedene Arten von Monstern und Helden geben. \\
\textbf{Priorität A} \\

\textbf{UR006} \\
\textbf{Aussage} Helden sollen verschiedene Klassen (bespielsweise Magier bzw. Krieger) annehmen können. \\
\textbf{Priorität B} \\

\textbf{UR007} \\
\textbf{Aussage} Die verschiedenen Heldenarten sollen sich in ihrem Aussehen unterscheiden. \\
\textbf{Priorität B} \\

\textbf{UR008} \\
\textbf{Aussage} Die verschiedenen Heldenarten sollen sich in ihren Stärken bzw. Fähigkeiten unterscheiden. \\
\textbf{Priorität C} \\

\textbf{UR009} \\
\textbf{Aussage} Der Benutzer soll seinen Helden individuell gestalten können bezüglich Haarfarbe, Kleidung... \\
\textbf{Priorität C} \\

\textbf{UR010} \\
\textbf{Aussage} Das Role-Play-Game soll bis zum 21.10.19 fertiggestellt und dem Kunden übergeben werden.\\
\textbf{Priorität A}

\newpage

\section{Pflichtenheft}
\label{Pflichtenheft}

\textbf{FR001} \\
\textbf{Aussage} Der Benutzer kann einen Helden frei in einer statischen zweidimensionalen Karte bewegen, auf die er von oben blickt. (siehe UR001/UR002) \\
\textbf{Priorität A} \\

\textbf{FR002} \\
\textbf{Aussage} Der Held kämpft gegen Monster. (siehe UR003) \\
\textbf{Priorität A} \\

\textbf{FR003} \\
\textbf{Aussage} Es gibt eine Kampfanimation. (siehe UR004) \\
\textbf{Priorität A} \\

\textbf{FR004} \\
\textbf{Aussage} Es soll eine zufallsbasierte Kartengeneration geben. \\
\textbf{Priorität B} \\

\textbf{FR005} \\
\textbf{Aussage} Es soll unterschiedliche Klassen (Magier und Krieger) geben, die sich in ihrem Aussehen und Fähigkeiten unterscheiden. (siehe UR006/UR007/UR008) \\
\textbf{Priorität B} \\

\textbf{FR006} \\
\textbf{Aussage} Es soll einen Startbildschirm geben, der den Benutzer begrüßt und einen \glqq{}Spielen\grqq{} - Knopf besitzt. (siehe UR001) \\
\textbf{Priorität C} \\

\textbf{FR007} \\
\textbf{Aussage} Vom Startbildschirm aus soll der Benutzer zu einer Art Hauptmenü - ein \glqq{}Hub\grqq{}, der ebenfalls einer Karte, auf die von oben herabgesehen, wird entspricht - gelangen. (siehe UR001) \\
\textbf{Priorität B} \\

\textbf{FR008} \\
\textbf{Aussage} Im Hauptmenü soll der Benutzer beispielsweise den Kampf gegen Monster starten können oder seinen Charakter wählen bzw. bearbeiten können. (siehe UR001/UR003/UR006/UR0099) \\
\textbf{Priorität B} \\

\textbf{FR009} \\
\textbf{Aussage} Die Erstellung eines Charakters durch den Benutzer (Geschlecht, Haarfarbe, Kleidung, Rasse) soll möglich sein. (siehe UR009) \\
\textbf{Priorität C} \\

\textbf{FR010} \\
\textbf{Aussage} Der Spielstand soll gespeichert werden. \\
\textbf{Priorität C} \\

\textbf{FR010} \\
\textbf{Aussage} Es soll ein Itemsystem geben, bei dem der Held ein Inventar besitzt, das er mit Items aus dem Monsterkampf füllen kann, die seine Attribute verändern. (siehe UR003) \\
\textbf{Priorität C} \\

\textbf{FR012} \\
\textbf{Aussage} Das Projekt soll bis zum 21.10.19 fertig gestellt werden. (siehe UR009) \\
\textbf{Priorität A} \\

\newpage

\section{Tagebuch}
\label{Tagebuch}

\myuline{Donnerstag, 19.09.19 (Tim):}

Nach Bekanntgeben des Beginns des Projekts, begann unsere Gruppe mit der Diskussion, wie wir alle zusammen an einem Projekt gleichzeitig arbeiten können. Dabei ging es sowohl um gemeinsame Java-Editoren für die tatsächliche Programmierarbeit als auch um gemeinsame Text-Editoren um eine einfach Arbeit am Tagebuch zu ermöglichen. Letztendlich wurde für Ersteres Git als Lösung vorgeschlagen und für Letzteres keine Lösung gefunden. \\
Des Weiteren wurden sich Gedanken gemacht über das grundlegende Aussehen der Anwendung, wobei sich auf eine \glqq{}Pixelgrafik\grqq{} geeinigt wurde. Die Ansicht des Kampfes sollte von der Seite sein - mit dem Monster in einer Ecke und dem Helden in der anderen. Lukas und Jonas äußerten Vorstellungen eines Hauptmenüs mit Optionen wie einem Shop u.Ä. \\

\myuline{Sonntag, 22.09.19 (Tim):}

Tim arbeitete weiter am Lastenheft. Es muss besprochen werden, ob der Rest der Gruppe mit dem erstellten Lastenheft und insbesondere mit den festgelegten Prioritäten übereinstimmt. \\

\myuline{Montag, 23.09.19 (Tim):}

Nach der Einführung in Git hat Lukas ein Repository auf GitHub erstellt und alle Gruppenmitglieder eingeladen. Dann wurden wenige grundlegende Ordner angelegt inklusive eines Ordners für die Dokumentation, in dem dieses Dokument abgespeichert wird, sodass alle \glqq{}gleichzeitig\grqq{} daran arbeiten können und Änderungen einfach eingesehen werden können. Somit wurde das Problem, für welches am 19.09. keine Lösung gefunde wurde, ebenfalls mit Git bzw. GitHub gelöst. \\

\myuline{Mittwoch, 25.09.19 (Tim):}

Tim bearbeitete die Titelseite dieses Dokuments. Die Gruppe hat sich zusammengesetzt und am Pflichtenheft gearbeitet. Es wurde diskutiert über die Umsetzung des Spiels und die vorherige Idee der Ansicht des Kampfes wurde ersetzt mit einer neuen (siehe Pflichtenheft). Danach wurde sich daran gemacht zu überlegen, wie die notwendige Arbeit am besten aufgeteilt werden kann. Zu jedem der Kernaspekte des Spiels, die zu implementieren sind, wurde ein \glqq{}Issue\grqq{} erstellt, in dem diskutiert werden soll, was bezüglich dieses Aspektes noch zu tun ist.\\
Im Anschluss wurde begonnen, ein Klassendiagramm anzufertigen um die bevorstehende Programmierarbeit zu erleichtern. Es wurde sich auf eine Struktur geeinigt, in der es folgende Klassen gibt: Eine \glqq{}Spielmanager\grqq{} - Klasse, eine Klasse, die die graphischen Operationen übernimmt und die abstrakten Klassen Held und Monster, die sich jeweils in weiteren Klassen spezifizieren. In der letzten Stunde vor den Ferien wurde (erneut) festgestellt, dass 2 der 5 Gruppenmitglieder fast die ganzen Ferien weg sind, was mehr Arbeit für die verbleibenden oder weniger Fortschritt bedeutet. \\

\myuline{Donnerstag, 25.09.19 (Lukas, Jonas):}

Lukas hat das graphische Grundgerüst implementiert, den Spieler mittels der WASD-Tasten bewegungsfähig gemacht und ihn als Bild dargestellt. \\
Jonas hat eine Skizze für den Hub erstellt und ein Testintergrundbild eingefügt. \\

\myuline{Freitag, 26.09.19 (Jonas):}

Jonas erstellte ein neues aktuelleres UML-Klassendiagramm.\\

\myuline{Freitag, 27.09.19 (Lukas):}

Lukas hat das Grundgerüst einer Map und eine erste Beispielsmap erstellt, dem Spieler eine Warte- und Laufanimation gegeben und mit dem Erstellen einer zufallsgenerierten Map begonnen. \\

\myuline{Samstag, 28.09.19 (Tim, Jonas, Lukas):}

Tim erstellte auf Lukas Grundlage der Helden-Klasse eine simple Monster-Klasse, die animiert ist, aber sonst nichts tut und noch keine Anwendung findet. \\
Jonas implementierte eine grafische Oberfläche für das Inventar und die Attribute des Helden. \\
Lukas vervollständigte die Zufallsgeneration einer Karte. Es muss noch an der graphisch korrekten Generation der Mauerteile gearbeitet werden. \\

\myuline{Sonntag, 29.09.19 (Tim, Lukas):}

Tim implementierte eine erste Schlaganimation des Helden. Es müssen noch \glqq{}unzulässige Eingaben\grqq{} des Benutzers abgefangen werden, da dieser zum jetzigen Zeitpunkt den Helden in der Schlaganimation festhalten kann. \\
Tim erstellte eine Methode zum Testen, ob sich ein gegebenes Monster mit dem Helden überlappt, welche die Basis für das kommende Kampfsystem legen soll.\\
Lukas hat weiter an der korrekten graphischen Generierung und Darstellung der Karte gearbeitet. \\

\myuline{Montag, 30.09.19 (Tim, Jonas, Lukas):}

Tim setzte ein eigenes Icon für das Fenster und versuchte sich daran, den Helden die Möglichkeit zu geben, dem Monster Schaden zufügen zu können, scheiterte aber. \\
Jonas implementierte einen neuen Helden, der nun auch nach links schauen kann und angreifen kann. \\
Lukas hat das Rendering räumlich korrekt gestaltet. Des weiteren extenden Monster und Spieler jetzt aus Entity um die Redundanz gemeinsamer Eigenschaften zu minimieren. Außerdem kann der Spieler einen Angriff nun nicht mehr abbrechen. \\

\myuline{Dienstag, 01.10.19 (Tim, Jonas, Lukas):}

Lukas verbesserte den Fehler in Tims \glqq{}Angriffscode\grqq{}, sodass der Held Monstern nun Schaden zufügen konnte. \\
Tim erstellte eine Methode, die zufällig Monster auf der Karte erzeugt und die Anzahl dieser in einer Variable mitzählt. Des Weiteren lässt er die Monster verschwinden, sobald ihre Lebenspunkte 0 sind, sie also tot sind. Tim erweiterte die zufällige Monstererstellung um eine Beeinflussung durch das Spielerlevel, sodas die Monster in ihrer Stärke an die des Spielers angepasst sind. Das Erfahrungssystem ist sonst noch nicht implementiert. \\
Jonas implementierte ein Inventarsystem. Nun können Items des Helden an und abgelegt werden. \\
Lukas hat ein Main- und Escape-Menu hinzugefügt. \\

\myuline{Mittwoch, 02.10.19 (Jonas):}

Jonas implementierte das Inventarsystem neu. Die Items sind nun eine eigene Klasse und werden nicht mehr als Integer gespeichert. Außerdem wurden die Bilder für die entsprechenden Items aus dem Internet ausgeschnitten und eingefügt. \\

\myuline{Donnerstag, 03.10.19 (Tim):}

Tim erweiterte das Pflichtenheft und stellte der Gruppe seine Vorstellung der wichtigsten Schritte vor, die seiner Meinung nach als nächstes erledigt werden müssen. Jonas stimmte diesen zu und ergänzte sie um eine Idee. \\

\myuline{Montag, 07.10.19 (Tim):}

Tim löste das Problem, dass die Monster mit zufälligen Attributen keine Werte zugewiesen bekommen. Er erweiterte den Setter der Lebenspunkte des Helden um eine Abfrage der maximalen Lebenspunkte, damit diese nicht überschritten werden. Er fügte eine visuelle Lebensleiste ein und positionierte den Text, der die Lebenspunkte angibt, in zentrierter Weise in der Lebensleiste. Die Farbänderung der Lebensanzeige sowie des Bildschirms hat er auf einen prozentualen Wert des maximalen Lebens, statt auf einen festen Lebenspunktewert gesetzt. Tim fügte ebenso eine Anzeige auf dem Display ein, die angibt, wie viele Monster noch am Leben sind. \\

\myuline{Dienstag, 08.10.19 (Tim):}

Tim erstellte einen \glqq{}Hub\grqq{}, von dem aus man in den Kampf mit Monstern gelangen kann. Dazu kopierte er die ExampleMap-Klasse, die zu Beginn von Lukas erstellt und dann von ihm mit der RandomMap-Klasse ersetzt wurde, und passte diese an die aktuellen Veränderungen des Map-Systems an, sodass der Spieler nicht über die Wände hinausgehen kann und die Map korrekt angezeigt wird. \\

\myuline{Mittwoch, 09.10.19 (Tim, Jonas):}

Tim erweiterte die Monster um verschiedene Monsterarten, die jeweils mit einer bestimmten Prozentzahl gespawnt werden. Des Weiteren beschränkte er die Erstellung von Monstern auf alles außer 1-Tile-breite Gänge, da die Methode zur Überprüfung von Kontakt zwischen Held und Monster dort seltsamerweise nicht funktionierte. \\
Jonas implementierte, dass nun auch ein Magier als Held gespielt werden kann, jedoch fehlt bei diesem noch die Animation des Projektils. \\

\myuline{Freitag, 11.10.19 (Tim, Jonas, Lukas):}

Tim begann mit der Implementation des Projektils und erstellte dazu eine neue Klasse \glqq{}Projectile\grqq{}. Er ermöglichte es, das ein Projektil erzeugt und dargestellt werden kann, jedoch fehlt beispielsweise noch, dass das Projektil Monstern schaden kann, dass es dorthin fliegt, wo es hinfliegen soll und dass es verschwindet, sobald es gegen eine Wand fliegt oder ein Monster getroffen hat. \\
Jonas machte es möglich, dass der Held nachdem er alle Monster getötet hat im Hub spawnt. Außerdem implementierte er eine Animation für ein Portal, dass den Spieler vom Hub zu den Monstern transportiert. \\
Lukas hat die Synchronität der Monsteranimationen entfernt um ihnen ein organischeres Verhalten zu geben und das graphische Rendering für alle Entities normiert um Redundanz zu verhindern.  \\

\myuline{Samstag, 12.10.19 (Tim):}

Tim arbeitete weiter an den Projektilen, sodass diese nun Monstern Schaden zufügen können, nicht über Wände und Monster hinausgehen und (sofern man stillsteht) dorthin fliegen, wo sie hin sollen. \\

\myuline{Sonntag, 13.10.19 (Tim, Jonas, Lukas):}

Tim stellte die Projektile vorerst fertig, sodass diese nun in jede Richtung (jede Diagonale sowie rechts/links und oben/unten) verschossen werden können und es keine Fehler mehr gibt, wenn diese eine Wand oder ein Monster treffen. Die Methode zur Überprüfung, ob das Projektil mit einem gegebenen Monster überlappt, funktionierte noch nicht exakt wie gedacht, muss also noch überarbeitet werden. \\
Jonas implementierte das Level-System, die zufallsbasierende Verteilung der Stats für die Items und das Updaten der Spieler Stats. \\
Lukas began mit dem Implementieren einer \glqq{}Monster verfolgen den Spieler\grqq{} Funktion mittels des A*-Algorithmuses. \\

\myuline{Montag, 14.10.19 (Tim, Lukas):}

Die Gruppe besprach, welche Aufgaben zu erledigen sind und verteilte diese auf die Mitglieder. Lukas wird sich noch um die Verbesserung der Monster- und Spielerbewegung kümmern, Philipp M. wird sich daran machen, dass Monster dem Spieler Schaden zufügen können, Philipp N. soll sich darum kümmern, dass der Benutzer den Heldentyp auswählen kann, Jonas wird sich weiter an das Inventar und Nutzen der Items setzen und Tim wird sich um die Vervollständigung der Projektile kümmern. \\
Lukas hat das begonne Feature weiterentwickelt. \\

\myuline{Dienstag, 15.10.19 (Philipp N.):}

Philipp N. beschließt, dass es dem Spiel noch an atmosphärischer Musik fehlt und kümmerte sich, nach Absprache mit Lukas, um die Organisation eines Komponisten für exklusive Musik. Diese wird, falls sie im Zeitraum fertiggestellt wird, am Ende in das Projekt eingefügt (da sie nicht rechtzeitig fertig geworden ist wurde dieses Projekt allerdings vernachlässigt). \\

\myuline{Mittwoch, 16.10.19 (Tim, Jonas, Lukas):}

Tim vollendete die Projektile, sodass eine Überlappung mit einem Monster nun korrekt erkannt wird, fügte eine Lebensleiste für Monster sowie einen \glqq{}Back to Hub\grqq{} - Knopf ein, der den Spieler aus dem Kampf wieder zurück in den Hub bringt. \\
Jonas fixte den Bug, dass die Angriffsanimation des Magiers nicht erschienen ist und löste das Problem, dass im Hub und in der eigentlichen Map unterschiedliche Spieler referenziert wurden und fügte diese zu einem zusammen. \\
Lukas hat das begonne Feature fertiggestellt und einen Debugging-Modus hinzugefügt. \\

\myuline{Donnerstag, 17.10.19 (Tim, Jonas):}

Tim löste das Problem nach dem Merge mit Lukas Branch, dass das Programm noch nicht an die andere X- und Y- Koordinatenspeicherung angepasst war, wie Lukas es verwendet hatte. Des Weiteren ermöglichte es, das Projektile nun auch vor Wänden entlang fliegen können. Er begann mit der Arbeit an durch den Benutzer auswählbaren Farben, in denen der Held dargestellt werden soll, sodass der Spieler seinen Charakter individualisieren kann. \\
Jonas erstellte eine Liste in welcher er detailliert 9 aktuelle Bugs beschreibt und machte kleine Gameplay Anpassungen.\\

\myuline{Freitag, 18.10.19 (Tim, Jonas):}

Tim löste eines der Probleme von Jonas' Liste, wegen welchem fast alle Projektile entfernt wurden, nachdem nur eines die Wand oder ein Monster getroffen hatte. Er bereitete ebenso weiter vor, dass der Benutzer den Helden sowie die Projektile in gewissen Farben auswählen kann. \\
Jonas erstellte eine eigene Klasse für das Inventar welche nun auch im Hub funktioniert und begann mit der Implementation eines Shopkeepers im Hub. Außerdem kann das Inventar jetzt auch von Ubuntu-Usern korrekt genutzt werden. \\

\myuline{Freitag, 18.10.19 (Tim, Jonas, Lukas):}

Jonas erstellte eine Levelbar, eine Goldanzeige und setzte die Implementation des Shopkeepers fort. Er implementierte eine Anzeige für die Stats der Items wenn der Spieler im Inventar mit der Maus über die Items fährt. \\
Lukas hat das Speichern und Laden des aktuellen Spielstandes hinzugefügt. \\

\myuline{Samstag, 19.10.19 (Jonas):}

Jonas erstellte eine Levelbar, eine Goldanzeige und setzte die Implementation des Shopkeepers fort. Er implementierte eine Anzeige für die Stats der Items wenn der Spieler im Inventar mit der Maus über die Items fährt. \\

\myuline{Samstag, 19.10.19 (Philipp N.):}

Philipp N. implementierte ein Menü, mit dem es möglich ist, den Helden zu wechseln und sorgt mit Hilfe von Lukas dafür, dass der Held letztendlich auch gewechselt wereden kann. \\

\myuline{Sonntag, 20.10.19 (Jonas, Tim, Lukas):}

Jonas implementierte den Shop fertig und er implementierte ein Portal, welches den Spieler nach dem Level zurück zum Hub bringt. Außerdem implementierte er das die von Tim ausgewählten Songs abgespielt werden können. Er löste das Problem, dass der Magier kein Projektil geschossen hat, wenn eine Bewegungstaste losgelassen wurde und sofort danach angegriffen wurde. Des Weiterem implementierte er eine Sterbeanimation.\\
Tim vollendete das Feature, das man die Farbe des Spielers und des Projektils auswählen kann und speicherte diese in einem Attribut, welches benutzt werden soll, um die ausgewählten Farben des Benutzers zu speichern. Ebenso löste er das Problem, das der Magier, wenn er viele Monster gleichzeitig getötet hat, selbst aus der Liste der Einheiten enfernt wird und somit unsichtbar wird. Er suchte zwei Musikstücke für den Hub und den Kampfbereich heraus, scheiterte jedoch daran, diese an den richtigen Stellen abspielen zu lassen. \\
Lukas hat hinzugefügt, dass Monster den Spieler angreifen können. Außerdem wurde die Speicher- und Ladefunktion um die nutzerspezifische Konfiguration der Farben erweitert. \\
Philipp M. implementierte dass die Monster den Spieler angreifen und töten können, und passte an dass der Spieler beim Level-UP Leben regeneriert.\\

\end{document}
