%BSP Bild
%\begin{figure}[h]
%\centering
%\includegraphics[width=7cm,height=2.8cm]{Fotos/logo.jpg}
%\end{figure}

%BSP Sektion
%\section{Test}
%\subsection{Umsetzung}
%\subsubsection{Eingabe}

%BSP Aufzählung
%\begin{enumerate}
%\item Wie realisieren wir den Roboter mechanisch?
%\item Wie lässt sich das Einlesen des Sudokus umsetzen?
%\item Wie implementieren wir den Algorithmus?
%\end{enumerate}

%BSP Aufzählung
%\begin{itemize}
%\item Helligkeitssensor
%\item Farbsensor
%\end{itemize}


%-------------------
%Beginn des Kopfbereiches
%-------------------

%Wir verwenden eine DIN-A4-Seite und die Schriftgrš§e 12.
\documentclass[a4paper,fontsize=13pt]{scrartcl} 

 \usepackage{float}

%ein Versuch die Silbentrennung auszuschalten
%\hyphenpenalty=10000
%\exhyphenpenalty=10000
\usepackage[none]{hyphenat} 
\sloppy

%Fuer Abbildungsverzeichnis
\usepackage{graphicx}

%Diese drei Pakete benštigen wir fŸr die Umlaute, Deutsche Silbentrennung etc.
\usepackage[utf8x]{inputenc}
\usepackage[ngerman]{babel}
\usepackage[T1]{fontenc}

%Für Zeilenabstände
\usepackage{setspace}

% Für Bilder 
\usepackage{graphicx}

%sorgt dafür, dass deutsche Anführungszeichen in Zitaten
\usepackage[babel,german=quotes]{csquotes}

%Das Paket erzeugt ein anklickbares Verzeichnis in der PDF-Datei.
\usepackage{hyperref}

\usepackage{titleref}

%für Eurosymbole
\usepackage{eurosym}

\usepackage{enumitem}


% Stil der Zitate und der Bibliographie
\bibliographystyle{unsrt}

% Einbindung von Quellcode
\usepackage{listings}  


%Das Paket wird fŸr die anderthalb-zeiligen Zeilenabstand benštigt
\usepackage{setspace}

%Fuer Abbildungsverz im Inhaltsverz
\usepackage{tocbibind}

\usepackage{pdfpages}

\usepackage{amssymb}
\usepackage{amsmath}
\usepackage{amsthm}

%Für Bögen über Strecken
\usepackage{arcs}

% Einbindung von Quellcode

\usepackage{xcolor}
\usepackage{listings}
\lstset{
	numbers=left, 
	numberstyle=\small, 
	numbersep=8pt,
	language=JAVA, 
	frame = single, 
	framexleftmargin=15pt}

%EinrŸckung eines neuen Absatzes
\setlength{\parindent}{0em}

%Definition der Ränder
\usepackage[paper=a4paper,left=30mm,right=30mm,top=30mm,bottom=30mm]{geometry} 

%Abstand der Fußnoten
%\deffootnote{1em}{1em}{\textsuperscript{\thefootnotemark\ }}

%Regeln, bis zu welcher Tiefe (section,subsection,subsubsection) †berschriften angezeigt werden sollen (Anzeige der †berschriften im Verzeichnis / Anzeige der Nummerierung)
%\setcounter{tocdepth}{3}
%\setcounter{secnumdepth}{3}

%-------------------
%Ende des Kopfbereiches
%-------------------

\usepackage{contour}
\usepackage{ulem}

\renewcommand{\ULdepth}{1.8pt}
\contourlength{0.8pt}

\newcommand{\myuline}[1]{%
  \uline{\phantom{#1}}%
  \llap{\contour{white}{#1}}%
}

\setcounter{tocdepth}{4}
\setcounter{secnumdepth}{4}

\begin{document}

\begin{titlepage}

\begin{center}


% Upper part of the page
\includegraphics[width=0.5\textwidth]{pictures/lmg8.png}\\[1.5cm]

%\textsc{\LARGE University of Beer}\\[1.5cm]

\textsc{\Large Dokumentation der Projektarbeit }\\[0.4cm]

\textsc{Leitung: Frau Müller}\\[0.4cm]


% Title
\noindent\rule{1\textwidth}{0.75pt}
%scheint die Linie zu machen
%es gibt medskip bigskip und smallskip
\vspace{0.4cm}


{ \huge \bfseries Role-Play-Game}\\[0.3cm]
\vspace{0.1cm}
\noindent\rule{1\textwidth}{0.75pt}
\vspace{0.4cm}


% Author and supervisor
\begin{minipage}{0.4\textwidth}
\begin{flushleft} \large


\end{flushleft}
\end{minipage}
\hfill
\begin{minipage}{0.5\textwidth}
\begin{flushright} \large
\vspace{1cm}
Lukas Fausten, Jonas Dalchow, Philipp Maleri, Philipp Neumer, Tim Holzenkamp \\
\vspace{0.5cm}
Informatik Leistungskurs 12 MLL\\[0.8cm]




\end{flushright}
\end{minipage}

\vfill

% Bottom of the page
{\large \today} 



\end{center}

\end{titlepage}

\newpage

\tableofcontents

\newpage

\section{Lastenheft}
\label{Lastenheft}

\textbf{UR001} \\
\textbf{Aussage} Das Role-Play-Game soll eine graphische Benutzeroberfläche haben. \\
\textbf{Priorität A} \\

\textbf{UR002} \\
\textbf{Aussage} Der Spieler soll sich einen Helden erstellen können. \\
\textbf{Priorität A} \\

\textbf{UR003} \\
\textbf{Aussage} Der erstellte Held soll in verschiedenen Runden gegen Monster kämpfen. \\
\textbf{Priorität A} \\

\textbf{UR004} \\
\textbf{Aussage} Es soll eine Kampfanimation geben. \\
\textbf{Priorität A} \\

\textbf{UR005} \\
\textbf{Aussage} Es soll verschiedene Arten von Monstern und Helden geben. \\
\textbf{Priorität A} \\

\textbf{UR006} \\
\textbf{Aussage} Helden sollen Magier bzw. Krieger sein können. \\
\textbf{Priorität B} \\

\textbf{UR007} \\
\textbf{Aussage} Die verschiedenen Heldenarten sollen sich in ihrem Aussehen unterscheiden. \\
\textbf{Priorität B} \\

\textbf{UR008} \\
\textbf{Aussage} Die verschiedenen Heldenarten sollen sich in ihren Stärken bzw. Fähigkeiten unterscheiden. \\
\textbf{Priorität C} \\

\textbf{UR009} \\
\textbf{Aussage} Der Benutzer soll seinen Helden individuell gestalten können bezüglich Haarfarbe, Kleidung... \\
\textbf{Priorität C} \\

\textbf{UR010} \\
\textbf{Aussage} Das Role-Play-Game soll bis zum 16.10 fertiggestellt und dem Kunden übergeben werden.\\
\textbf{Priorität A}

\newpage

\section{Pflichtenheft}
\label{Pflichtenheft}

\newpage

\section{Tagebuch}
\label{Tagebuch}

\myuline{Donnerstag, 19.09.19:}

Nach Bekanntgeben des Beginns des Projekts, begann unsere Gruppe mit der Diskussion, wie wir alle zusammen an einem Projekt gleichzeitig arbeiten können. Dabei ging es sowohl um gemeinsame Java-Editoren für die tatsächliche Programmierarbeit als auch um gemeinsame Text-Editoren um eine einfach Arbeit am Tagebuch zu ermöglichen. Letztendlich wurde für Ersteres Git als Lösung vorgeschlagen und für Letzteres keine Lösung gefunden. \\
Des Weiteren wurden sich Gedanken gemacht über das grundlegende Aussehen der Anwendung, wobei sich auf eine \glqq{}Pixelgrafik\grqq{} geeinigt wurde. Die Ansicht des Kampfes sollte von der Seite sein - mit dem Monster in einer Ecke und dem Helden in der anderen. Lukas und Jonas äußerten Vorstellungen eines Hauptmenüs mit Optionen wie einem Shop u.Ä. \\

\myuline{Sonntag, 22.09.19:}

Tim arbeitete weiter am Lastenheft und am vorherigen Tagebucheintrag. Es muss besprochen werden, ob der Rest der Gruppe mit dem erstellten Lastenheft und insbesondere mit den festgelegten Prioritäten übereinstimmt. \\

\myuline{Montag, 23.09.19:}

Nach der Einführung in Git hat Lukas ein Projekt auf GitHub erstellt und alle Gruppenmitglieder eingeladen. Dann wurden wenige grundlegende Ordner angelegt inklusive eines Ordners für die Dokumentation, in dem dieses Dokument abgespeichert wird, sodass alle \glqq{}gleichzeitig\grqq{} daran arbeiten können und Änderungen einfach eingesehen werden können. Somit wurde das Problem, für welches am 19.09. keine Lösung gefunde wurde, ebenfalls mit Git bzw. GitHub gelöst. \\

\myuline{Mittwoch, 25.09.19:}

Tim bearbeitete die Titelseite dieses Dokuments. \\

\end{document}
